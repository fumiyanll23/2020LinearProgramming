%%% 中村研ゼミ第2回(2020/5/28実施)
%%% 使用テキスト: 田村明久、村松正和『最適化法』(共立出版)

\documentclass[unicode, 12pt, aspectratio = 43]{beamer}
\usepackage{bxdpx-beamer}

%%% テーマの指定、省略時は default になる
\usetheme[progressbar = frametitle]{metropolis}
%\usepackage{zxjatype}
%\setCJKmainfont[Scale=0.95]{Meiryo}
\usepackage{bm}
\usepackage{color}
%\usepackage{listings,jlisting}
%\lstset{language={C}, basicstyle=\ttfamily\footnotesize,
%	commentstyle=\textit, classoffset=1, frame=tRBl, framesep=5pt,
%	numbers=left, stepnumber=1, numberstyle=\footnotesize, tabsize=2}
%\usepackage{slashbox}
%\usepackage{hyperref}
\usepackage{ascmac}
\usepackage{amsmath}
\usepackage{amsfonts}

%\setsansfont[
%	BoldFont={Fira Sans SemiBold},
%	ItalicFont={Fira Sans Italic},
%	BoldItalicFont={Fira Sans SemiBold Italic}
%	]{Fira Sans}
	
%%%  日本語フォントをゴシックに、数式フォントを太字に変更する
\renewcommand{\kanjifamilydefault}{\gtdefault}
\mathversion{bold}

%%% 著者の情報など
\title{第2回}
\subtitle{標準形と双対問題} % 省略可
\author{171-T7710 成田 史弥}
\institute{工学部 数理工学科 城本研究室}
\date{} % 省略すると自動挿入

%%% 以下、スライド
\begin{document}

%%% cover page
\begin{frame}
	\titlepage
\end{frame}

\begin{frame}[fragile]
	\frametitle{発表内容}
	\begin{enumerate}
		\item (Remember)線形計画問題 % テキストの該当範囲も提示
		\item 不等式標準形
		\item 等式標準形
		\item 双対問題
	\end{enumerate}
\end{frame}

\begin{frame}[fragile]
	\frametitle{(Remember)線形計画問題}
	\begin{itembox}[l]{Def. 1.2.a(線形計画問題)}
		\begin{itemize}
			\item 実定数 $a_{ij}, b_{i}, c_{j}(i = 1, \dots, m; j = 1, \dots, n)$
			\item 実変数 $x_{j}(j = 1, \dots, n)$
			\item $0 \le \ell \le m$
		\end{itemize}
		最小化 \ $c_1x_1+ \dots+ c_nx_n$ \\
		条件 \ $a_{i1}x_1 + \dots+ a_{in}x_n \le b_i \ (i = 1, \dots, \ell)$ \\
		\   $a_{i1}x_1 + \dots+ a_{in}x_n = b_i \ (i = \ell+1, \dots, m)$
	\end{itembox}
\end{frame}

\begin{frame}[fragile]
	\frametitle{(Remember)線形計画問題}
	\begin{itembox}[l]{Def. 1.2.a(線形計画問題)}
		\begin{itemize}
			\item 実定数 $a_{ij}, b_i, c_j(i = 1, \dots, m; j = 1, \dots, n)$
			\item 実変数 $x_j(j = 1, \dots, n)$
			\item $0 \le \ell \le m$
		\end{itemize}
		最小化 \ $c_1x_1+ \dots+ c_nx_n$ \\
		条件 \ $a_{i1}x_1 + \dots+ a_{in}x_n \le b_i \ (i = 1, \dots, \ell)$ \\
		\   $a_{i1}x_1 + \dots+ a_{in}x_n = b_i \ (i = \ell+1, \dots, m)$
	\end{itembox}
	\begin{itemize}
		\item 目的関数も制約式も1次式(線形)である
		\item 制約式の本数が有限である
		\item 制約不等式が等号つきである
	\end{itemize}
\end{frame}

\begin{frame}[fragile]
	\frametitle{不等式標準形}
	\begin{itembox}[l]{Def. 2.1.a(不等式標準形)}
		\begin{itemize}
			\item 実定数 $a_{ij}, b_i, c_j(i = 1, \dots, m; j = 1, \dots, n)$
			\item 実変数 $x_j(j = 1, \dots, n)$
			\item $0 \le \ell \le m$
		\end{itemize}
		最小化 \ $c_1x_1+ \dots+ c_nx_n$ \\
		条件 \ $a_{i1}x_1 + \dots+ a_{in}x_n \ge b_i \ (i = 1, \dots, m)$ \\
		\   $x_j \ge 0 \ (j = 1, \dots, n)$
	\end{itembox}
\end{frame}

\begin{frame}[fragile]
	\frametitle{不等式標準形}
	\begin{itembox}[l]{Def. 2.1.a(不等式標準形)}
		\begin{itemize}
			\item 行列
\[
  A = \left(
    \begin{array}{cccc}
      a_{11} & \ldots & a_{1n} \\
      \vdots & \ddots & \vdots \\
      a_{m1} & \ldots & a_{mn}
    \end{array}
  \right)
\]	
			\item ベクトル ${\bf b} = (b_1, \dots, b_m)^T$
			\item ベクトル ${\bf c} = (c_1, \dots, c_m)^T$
			\item ベクトル ${\bf x} = (x_1, \dots, x_m)^T$
		\end{itemize}
		
		最小化 \ ${\bf c}^T {\bf x}$ \\
		条件 \ $A{\bf x} \ge {\bf b}, {\bf x} \ge {\bf 0}$
	\end{itembox}
\end{frame}

\begin{frame}[fragile]
	\frametitle{不等式標準形}
	\begin{itembox}[l]{Def. 2.1.a(不等式標準形)}
		最小化 \ ${\bf c}^T {\bf x}$ \\
		条件 \ $A{\bf x} \ge {\bf b}, {\bf x} \ge {\bf 0}$
	\end{itembox}
	\begin{itemize}
		\item 最小化問題である
		\item 制約式が全て(左辺が大の)不等式である
		\item 全ての変数が非負である
	\end{itemize}
\end{frame}

\begin{frame}[fragile]
	\frametitle{不等式標準形}
	e.g.2.1) 次の線形計画問題は不等式標準形である: \\
	最小化 \ $-2x_1 - x_2 - x_3$ \\
	条件 \ $-2x_1 - 2x_2 + x_3 \ge -4$ \\
	\   $-2x_1 - 4x_3 \ge -4$ \\
	\   $4x_1 - 3x_2 + x_3 \ge -1$ \\
	\   $x_1 \ge 0, x_2 \ge 0, x_3 \ge 0.$
\end{frame}

\begin{frame}[fragile]
	\frametitle{不等式標準形}
	\begin{itembox}[l]{Prop. 2.1}
		任意の線形計画問題は不等式標準形に変換できる.
	\end{itembox}
\end{frame}

\begin{frame}[fragile]
	\frametitle{不等式標準形}
	\begin{itemize}
		\item 式の同値変形 \\
		$\displaystyle \sum_{j=1}^n a_j x_j = b$
		→$\displaystyle \sum_{j=1}^n a_j x_j \le b, \sum_{j=1}^n a_j x_j \ge b.$
		\item $-1$倍 \\
		最大化 $\displaystyle \sum_{j=1}^n c_j x_j$ → 最小化 $\displaystyle -\sum_{j=1}^n c_j x_j, $ \\
		$\displaystyle \sum_{j=1}^n a_j x_j \le b$ → $\displaystyle -\sum_{j=1}^n a_j x_j \ge -b.$
		\item 差による表現 \\
		変数$x$(非負制約なし) \\
		→$x = x_1 - x_2, x_1 \ge 0, x_2 \ge 0.$
	\end{itemize}
\end{frame}

\begin{frame}[fragile]
	\frametitle{等式標準形}
	\begin{itembox}[l]{Def. 2.1.b(等式標準形)}
		最小化 \ $c_1x_1+ \dots+ c_nx_n$ \\
		条件 \ $a_{i1}x_1 + \dots+ a_{in}x_n = b_i \ (i = 1, \dots, m)$ \\
		\   $x_j \ge 0 \ (j = 1, \dots, n)$
	\end{itembox}
\end{frame}

\begin{frame}[fragile]
	\frametitle{等式標準形}
	\begin{itembox}[l]{Def. 2.1.b(等式標準形)}
		最小化 \ ${\bf c}^T {\bf x}$ \\
		条件 \ $A{\bf x} = {\bf b}, {\bf x} \ge {\bf 0}$
	\end{itembox}
\end{frame}

\begin{frame}[fragile]
	\frametitle{等式標準形}
	\begin{itembox}[l]{Prop. 2.2}
		任意の線形計画問題は等式標準形に変換できる.
	\end{itembox}
\end{frame}

\begin{frame}[fragile]
	\frametitle{等式標準形}
	e.g.2.2) e.g.2.1の不等式標準形は, 次のように等式標準形に変換される: \\
	最小化 \ $-2x_1 - x_2 - x_3$ \\
	条件 \ $-2x_1 - 2x_2 + x_3 - x_4 = -4$ \\
	\   $-2x_1 - 4x_3  - x_5 = -4$ \\
	\   $4x_1 - 3x_2 + x_3 - x_6 = -1$ \\
	\   $x_1, x_2, x_3, x_4, x_5, x_6 \ge 0.$
\end{frame}

\begin{frame}[fragile]
	\frametitle{双対問題}
	e.g.2.1: \\
	最小化 \ $-2x_1 - x_2 - x_3$ \\
	条件 \ $-2x_1 - 2x_2 + x_3 \ge -4$ \\
	\   $-2x_1 - 4x_3 \ge -4$ \\
	\   $4x_1 - 3x_2 + x_3 \ge -1$ \\
	\   $x_1 \ge 0, x_2 \ge 0, x_3 \ge 0.$
\end{frame}

\begin{frame}[fragile]
	\frametitle{双対問題}
	e.g.2.1の双対問題: \\
	最大化 \ $-4y_1 - 4y_2 - y_3$ \\
	条件 \ $-2y_1 - 2y_2 + 4y_3 \le -2$ \\
	\   $-2y_1 - 3y_3 \le -1$ \\
	\   $y_1 - 4y_2 + y_3 \le -1$ \\
	\   $y_1 \ge 0, y_2 \ge 0, y_3 \ge 0.$
\end{frame}

\begin{frame}[fragile]
	\frametitle{双対問題}
	\begin{itembox}[l]{Def. 2.2.a(主問題)}
		最小化 \ ${\bf c}^T {\bf x}$ \\
		条件 \ $A{\bf x} \ge {\bf b}, {\bf x} \ge {\bf 0}$
	\end{itembox}
	\begin{itembox}[l]{Def. 2.2.b(双対問題)}
	\begin{itemize}
		\item 変数ベクトル ${\bf y} \in \mathbb{R}^m$
	\end{itemize}
		最大化 \ ${\bf b}^T {\bf y}$ \\
		条件 \ $A^T{\bf y} \le {\bf c}, {\bf y} \ge {\bf 0}$
	\end{itembox}
\end{frame}

\begin{frame}[fragile]
	\frametitle{双対問題}
	e.g.2.1)先の考え方を応用して, 等式標準形の双対問題について考察せよ.
\end{frame}

\begin{frame}[fragile]
	\frametitle{双対問題}
	等式標準形: \\
	最小化 \ ${\bf c}^T {\bf x}$ \\
	条件 \ $A{\bf x} = {\bf b}, {\bf x} \ge {\bf 0}.$
\end{frame}

\begin{frame}[fragile]
	\frametitle{双対問題}
	等式標準形: \\
	最小化 \ ${\bf c}^T {\bf x}$ \\
	条件 \ $A{\bf x} = {\bf b}, {\bf x} \ge {\bf 0}.$ \\ \vspace{1em}
	等式標準形の双対問題: \\
	最大化 \ ${\bf b}^T {\bf y}$ \\
	条件 \ $A^T{\bf y} \le {\bf c}.$
\end{frame}

%\begin{frame}[fragile]
%	\frametitle{参考文献}
%\end{frame}

\end{document}